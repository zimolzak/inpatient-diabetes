\documentclass{tufte-handout}
% \documentclass{article}
\title{Type 2 diabetes management for adult inpatients}
\author{Andrew J.\ Zimolzak, MD, MMSc}
\date{July 17, 2015}

\begin{document}

\maketitle

~\\ % makes it look better in tufte-handout class

Why should we care about glucose in the inpatient setting in the first
place? There are plenty of papers on ``tight glycemic control'' in ICU
patients and in MI patients, but we won't go in to those in
detail. There are studies of surgical ward patients and stroke ward
patients too. There is a \emph{retrospective} study of hospitalized
patients that showed that hyperglycemia is associated with higher
mortality risk.\footnote{I did not read this paper, but the reference
  is: Umpierrez GE \emph{et al.} J Clin Endocrinol Metab 87:978--982,
  2002.}

Why does sliding scale alone not make sense? Sugars are checked
pre-meal. If the pre-lunch sugar is 250, what does it mean? It means
maybe the patient ate a big breakfast and didn't have enough insulin
on board at breakfast to compensate---in other words it's an effect
of something that happened in the past. There is nothing you can do in
the present to change what happened in the past.

If you give insulin to this patient, things might go OK. If he eats
very little lunch, though, he might become \emph{hypo}glycemic. High
sugar pre-lunch means we \emph{should have} given more insulin with
breakfast, not that we should give more insulin now. I suspect that
hypoglycemia due to sliding scale alone results in providers labeling
patients as ``brittle diabetics,'' making future providers hesitant to
do anything to treat their diabetes.

\section{What the guidelines say}

Subcutaneous insulin ``may be the most practical method'' for non-ICU
patients. ``In stable patients who are eating, orally administered
agents may be used,'' but are not appropriate in other
circumstances. ``The traditional regular insulin `sliding scale' is
not recommended\ldots{}. This `retrospective' form of insulin
replacement is inherently illogical\ldots{}.'' Basal plus short or
rapid acting prandial is recommended. ``Correction insulin'' is often
used too.\footnote{ACE/ADA Task Force on Inpatient Diabetes. Consensus
  statement on inpatient diabetes and glycemic control. Diabetes
  Care. 2006 Aug;29(8):1955--62.} Also, please note that premeal sugar
checks are for determining correction insulin and helping you make
changes to future dosing. Pre-\emph{bedtime} glucose check is to make
sure that the glucose is not too low (and also to help change future
dosing)---this evening check is usually \emph{not} meant as an
opportunity to give more insulin.

One guideline says there are no RCTs of ``intensive glycemic control''
in non-ICU hospitalized patients, so ``based on clinical experience
and judgment,'' for noncritically ill patients treated with insulin,
premeal glucose should be <140 mg/dL, and random (postprandial)
glucose should be <180. Consider easing up if numbers are <100, and
definitely change if numbers <70.\footnote{AACE/ADA Consensus
  Statement on Inpatient Glycemic Control. Endocr Pract
  2009;15(4):1--17}

\section{Original studies of non-ICU patients}

There is an RCT\footnote{Umpierrez GE \emph{et al.} RABBIT 2
  trial. Diabetes Care 2007;30(9):2181--6} of basal-bolus versus
sliding-scale in 130 non-ICU patients. Patients had a known diagnoses
of diabetes, were insulin-naive, and were on general medicine
services. A typical patient would be 56 years old, African-American,
A1c 8.8\%, with a 5-day hospital stay. In the basal-bolus group, 66\%
met the goal of <140, whereas in the sliding-scale group, 33\% met the
goal. Secondary endpoints of hypoglycemia and LOS were the same. The
regimen was total daily dose = 0.4 units/kg for glucose 140--200, or
0.5 units/kg for glucose >200, with half of total daily dose given as
glargine, and half as glulisine divided TID (think 14 + 5 + 5 + 5, for
a 70 kg person). There was also correction dose, and protocol for up-
or down-titrating the glargine and glulisine. The outcome of this
study was meeting the glycemic goal. A bigger study is needed to look
at ``harder'' outcomes. A later study of about 300 patients compared
basal-bolus-correction with basal-bolus and
sliding-scale\footnote{Umpierrez \emph{et al.}  Diabetes
  Care. 2013;36(8):2169--74.} with similar outcomes. Finally, a trial
of 200 med/surg ward patients compared basal-bolus with bedtime
correction insulin, versus basal-bolus without bedtime
correction. There was no difference in glycemic
control.\footnote{Vellanki P \emph{et al.} Diabetes
  Care. 2015;38(4):568--74.}

\section{Practical notes}

In practice this will be aspart 5 units TID, glargine 14 units daily
at 17:00, for a 70 kg person. I am OK if you use regular+NPH instead of
aspart+glargine. I am also OK if you want to start people low (0.3
units/kg instead of 0.4), but doses obviously may need to be adjusted
on future hospital days depending on the level of glycemic control. It
is \emph{very important} to adjust this regimen for NPO patients or
anyone not taking three normal meals daily. The general idea for NPO
patients is: stop the prandial and correction dose, cut the basal in
half, and give maintenance fluids with a little dextrose, but this
brief description may not be detailed enough if you've never done this
adjustment before.

\end{document}

% LocalWords:  pre didn glycemic Umpierrez et al Clin Endocrinol Metab prandial
% LocalWords:  RCTs noncritically premeal dL AACE Endocr Pract NPO RCT A1c LOS
% LocalWords:  glargine glulisine TID titrating surg Vellanki aspart NPH ve
