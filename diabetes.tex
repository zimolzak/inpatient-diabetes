% \documentclass{tufte-handout}
\documentclass{article}
\title{Type 2 diabetes management for adult inpatients}
\author{Andrew J.\ Zimolzak, MD, MMSc}
\date{July 17, 2015}

\begin{document}

\maketitle

% ~\\ % makes it look better in tufte-handout class

Why does sliding scale alone not make sense? Sugars are checked
pre-meal. If the pre-lunch sugar is 250, what does it mean? It means
maybe the patient ate a big breakfast and didn't have enough insulin
on board at breakfast to compensate---in other words it's an effect
of something that happened in the past. There is nothing you can do in
the present to change what happened in the past.

If you give insulin to this patient, things might go OK. If he eats
very little lunch, though, he might become
\emph{hypo}glycemic. High sugar pre-lunch means we \emph{should have}
given more insulin with breakfast.

There are plenty of papers on ``tight glycemic control'' in ICU
patients and in MI patients, but we won't go in to those in
detail. There are studies of surgical ward patients and stroke ward
patients too. There is a \emph{retrospective} study of hospitalized
patients that showed that hyperglycemia is associated with higher
mortality risk.\footnote{I did not read this paper, but the reference
  is: Umpierrez GE \emph{et al.} J Clin Endocrinol Metab 87:978--982,
  2002.}

Subcutaneous insulin ``may be the most practical method'' for non-ICU
patients. ``In stable patients who are eating, orally administered
agents may be used,'' but are not appropriate in other
circumstances. ``The traditional regular insulin `sliding scale' is
not recommended\ldots{}. This `retrospective' form of insulin
replacement is inherently illogical\ldots{}.'' Basal plus short or
rapid acting prandial is recommended. ``Correction insulin'' is often
used too.\footnote{ACE/ADA Task Force on Inpatient Diabetes. Consensus
  statement on inpatient diabetes and glycemic control. Diabetes
  Care. 2006 Aug;29(8):1955--62.}

There are no RCTs of ``intensive glycemic control'' in non-ICU
hospitalized patients. Based on clinical experience and judgment, for
noncritically ill patients treated with insulin, premeal glucose
should be <140 mg/dL, and random (postprandial) glucose should be
<180. Consider easing up if numbers are <100, and definitely change if
numbers <70.\footnote{AACE/ADA Consensus Statement on Inpatient
  Glycemic Control. Endocr Pract 2009;15(4):1--17}

In brief, for NPO patients, stop the prandial and correction dose, cut
the basal in half, and give maintenance fluids with a little
dextrose.

\end{document}

% LocalWords:  pre didn glycemic Umpierrez et al Clin Endocrinol Metab prandial
% LocalWords:  RCTs noncritically premeal dL AACE Endocr Pract NPO
